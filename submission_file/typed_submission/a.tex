\documentclass{article}
\title{Energy Cloud Exam 1}
\author{Isidor Heo}

\begin{document}
\maketitle 

\section*{Problem 1}

Renewable energy technologies play a critical role in future energy systems due to their sustainability and 'cost-competitiveness'. As the costs of solar and wind power continue to decrease over various developments in optimizing the engineering of these sites/plants as well as energy storage methods, renewables are becoming more and more accessible and economically viable. It is especially considered significant addition to the classical energy production methods that are evolving to include renewables rather than choosing entirely one or the other: hybrid methods of incorporating renewable energy systems emphasizes the effort for sustainability and energy efficiency. However, these sorts of integration to an already-well-established power plants/systems can pose logistical challenges.

It also poses the advantage for energy independence and security. Some challenges are: intermittency, grid integration, and policy barriers. Overcoming these obstacles require investments in conducive infrastructure, consistent technological upgrades, and voicing for supportive policies to accelerate the transition to renewable energy-based energy production system used worldwide, even in developing countries, and achieve a sustainable energy future.

% \section*{Problem 2}

% Energy is obtained from the pressure drop of water from 8 bars to 1 bar, the atmospheric, ambient pressure. $$ q - w = \Delta h + \Delta \frac{v^2}{2} + \Delta gz \Rightarrow w = \Delta h $$ Note that no heat is transferred into the system. $$ \Delta h = \Delta ( u + pv ) = (pv)_f - (pv)_i \Rightarrow \Delta h = \Delta (pv) = \Delta (\frac{p}{\rho}) $$

% Since both pressure drop and density are given, substitution yields $ \Delta h = 700 $ Nm/kg. By multiplying the mass flow rate of 1 kg/s, we obtain 700 J/s. The turbine and generator efficiencies can be multiplied to obtain $ 700 \times 0.80 \times 0.90 = 504 $ J/s.

% [REFERENCE IMAGE IN PAGE 4 OF PDF]


% \section*{Problem 3}

% \section*{Problem 4}

% $$ \eta = \tau \alpha - U \frac{T_c - T_a }{G } = (0.86)(0.95) - \left(3 \frac{\rm W}{\rm m^2 \cdot K }\right) \frac{(45 - 23) \rm K }{750 \mathrm{W/m^2}} = 0.729 $$

% \section*{Problem 5}

% $$ R_{tot} = R_w + R_i + R_g = \frac{L_w }{\lambda_w } + \frac{L_i }{\lambda_i } + \frac{L_g }{\lambda_g } $$ $$ = \left(\frac{50}{0.002} + \frac{200}{0.04} + \frac{10}{0.5}\right) \frac{\rm mm \cdot m \cdot K }{\rm W } \left( \frac{1 \rm m}{10^3 \rm mm} \right) = 7.52 \frac{\rm m^2 K }{\rm W } $$

% $$ U = R^{-1} = 0.133 \frac{\rm W }{\rm m^2 \cdot K } $$ But adding $U$ for inside and outside convection and radiation yields $$U_{tot} = (0.133 + 15 + 10) \frac{\rm W }{\rm m^2 \cdot K } = 25.133 \frac{\rm W }{\rm m^2 \cdot K } $$

% Given $ Q/A = U \Delta T$, $$ \frac{Q}{A} = \left(25.133 \frac{\rm W }{\rm m^2 \cdot K }\right) (33 \rm K) = 829 \frac{\rm W }{\rm m^2} $$

\section*{Problem 10}

\begin{enumerate}
    \item Solar Energy 
    \begin{enumerate}
        \item Abundant in nature, relatively low overheads post-installation, adaptable in size (can be installed on a house or in a solar farm)
        \item highly weather-dependent, solar farms require large lands with the perfect climate, energy storage needs to be implemented alongside
    \end{enumerate}
    \item Wind Energy
    \begin{enumerate}
        \item Abundant in nature, installation can be done onshore or offshore, low maintenance post-installation, widely known and used 
        \item wind speed can vary a lot, can drive out residential or commercial areas nearby due to its noise, kills birds, and costly to install 
    \end{enumerate}
    \item Hydropower 
    \begin{enumerate}
        \item more or less reliable and predictable, long life span for hydroplants, can be used for flood control or irrigation as well 
        \item dam construction is invasive, sediments build up over time, aquatic wildlife can be disturbed
    \end{enumerate}
    \item Biomass Energy
    \begin{enumerate}
        \item can be originated from many different organic sources (agricultural residues, wood, organic waste) and hence helps reduces waste and emissions, carbon-neutral when managed well
        \item biomass burning may cause carbon emission and air pollution, needs to be managed carefully so as not to damage the ecosystem or cause deforestation 
    \end{enumerate}
    \item Geothermal Energy 
    \begin{enumerate}
        \item reliable and constant, can be used for both electricity generation and heating, almost no greenhouse gas emission, relatively small 'land footprint'
        \item useful geothermal reservoirs are limited, startup costs of selecting a site and drilling are high, unexpected seismic activities can interfere or even be caused by the development
    \end{enumerate}
\end{enumerate}

\end{document}